\begin{center}
	\large Выводы
\end{center}

\addcontentsline{toc}{section}{Выводы}

1. На основе методик машинного обучения разработан метод единовременного считывания квантового состояния сверхпроводящих кубитов; 

2. Разработана методика разделения ошибок приготовления и считывания квантовых состояний сверхпроводящих кубитов путём использования последовательности двух считывающих импульсов на основное состояние;

3. Создан программно-аппаратный комплекс, позволяющий проводить оцифровку и вычисление статистических параметров сигнала за сверхмалые времена (порядка сотен наносекунд);

4. На базе данного программно-аппаратного комплекса была реализована схема обратной связи для коррекции начального состояния сверхпроводящих кубитов и подготовки их к проведению квантовых алгоритмов.
