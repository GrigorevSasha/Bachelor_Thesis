
\begin{center}
	\large Аннотация
\end{center}



Выполнение любых алгоритмов подразумевает строгий контроль входных параметров. Не являются исключениями и алгоритмы, основанные на возможности некоторой физической системы находиться в когерентной суперпозиции двух собственных состояний. В данной дипломной работе описан процесс разработки программно-аппаратного комплекса по эффективной корректировке исходных состояний сверхпроводящих кубитов. Созданное устройство, используя алгоритмы машинного обучения, считывает и распознаёт текущее квантовое состояние и при необходимости переводит его в нужное. Весь цикл занимает время, существенно меньшее времён декогеренции и релаксации.

Выпускная квалификационная работа изложена на 40 страницах, содержит 33 рисунка, список использованных источников из 28 наименований. 