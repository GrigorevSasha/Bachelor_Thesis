\begin{center}
	\large{Введение}
\end{center}
\addcontentsline{toc}{section}{Введение}%\dotfill}

В 1982 году Ричард Филлипс Фейнман заметил, что моделирование даже самой простой квантово-механической системы требует колоссальных вычислительных ресурсов обычного классического компьютера, и высказал идею о том, что для решения таких задач можно использовать вычислительные системы, основанные на принципах квантовой механики  \cite{Feynman1982}.

Своё продолжение эта идея нашла в работах американского математика и информатика Питера Шора в 1995 году \cite{Shor1995}. Был разработан так называемый алгоритм Шора, решающий экспоненциально сложную с точки зрения подхода классического компьютера задачу факторизации больших чисел (разбиения их на простые сомножители). Эта задача нисколько не является абстрактной. На невозможности классического компьютера факторизовывать большие числа современная криптография построила почти всю систему защиты информации.

Ещё одно интересное применение квантовых принципов вычисления было найдено в виде  так называемого алгоритма Гровера (GSA- grover search algorithm \cite{Grover1996}). Это решение задачи подбора, т.е. решение уравнения вида $f(x_1,x_2 ... x_n)=1$, где $f(x_1,x_2 ... x_n)$ есть некоторая булева функция многих переменных. Проще говоря, нужно подобрать такую комбинацию  $\{x_i\}$, для которой условие выполняется. Классический компьютер требует последовательного перебора всех $N=2^n$ возможных вариантов и, соотвественно, решает эту задачу за $2^n$ временных единиц. Используя же принципы квантовой механики GSA, решает эту задачу за $\frac{\pi}{4}\sqrt{2^n}$ временных единиц. 

Всё это показывает, что создание универсального квантового компьютера позволит наиболее рационально подойти к целому ряду вопросов, связанных с моделированием сложных систем, защитой и контролем информации, подбором включений для создания материалов с заданными свойствами, подбором компонентов лекарств и т.д. 

Очевидно, что для корректного выполнения таких задач необходимо уметь приготавливать квантовые состояния, выполнять на них логические операции и считывать результат. Этой теме и посвящена данная работа. 
